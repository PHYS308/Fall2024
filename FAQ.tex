% Options for packages loaded elsewhere
\PassOptionsToPackage{unicode}{hyperref}
\PassOptionsToPackage{hyphens}{url}
\PassOptionsToPackage{dvipsnames,svgnames,x11names}{xcolor}
%
\documentclass[
  letterpaper,
  DIV=11,
  numbers=noendperiod]{scrartcl}

\usepackage{amsmath,amssymb}
\usepackage{iftex}
\ifPDFTeX
  \usepackage[T1]{fontenc}
  \usepackage[utf8]{inputenc}
  \usepackage{textcomp} % provide euro and other symbols
\else % if luatex or xetex
  \usepackage{unicode-math}
  \defaultfontfeatures{Scale=MatchLowercase}
  \defaultfontfeatures[\rmfamily]{Ligatures=TeX,Scale=1}
\fi
\usepackage{lmodern}
\ifPDFTeX\else  
    % xetex/luatex font selection
\fi
% Use upquote if available, for straight quotes in verbatim environments
\IfFileExists{upquote.sty}{\usepackage{upquote}}{}
\IfFileExists{microtype.sty}{% use microtype if available
  \usepackage[]{microtype}
  \UseMicrotypeSet[protrusion]{basicmath} % disable protrusion for tt fonts
}{}
\makeatletter
\@ifundefined{KOMAClassName}{% if non-KOMA class
  \IfFileExists{parskip.sty}{%
    \usepackage{parskip}
  }{% else
    \setlength{\parindent}{0pt}
    \setlength{\parskip}{6pt plus 2pt minus 1pt}}
}{% if KOMA class
  \KOMAoptions{parskip=half}}
\makeatother
\usepackage{xcolor}
\setlength{\emergencystretch}{3em} % prevent overfull lines
\setcounter{secnumdepth}{5}
% Make \paragraph and \subparagraph free-standing
\ifx\paragraph\undefined\else
  \let\oldparagraph\paragraph
  \renewcommand{\paragraph}[1]{\oldparagraph{#1}\mbox{}}
\fi
\ifx\subparagraph\undefined\else
  \let\oldsubparagraph\subparagraph
  \renewcommand{\subparagraph}[1]{\oldsubparagraph{#1}\mbox{}}
\fi


\providecommand{\tightlist}{%
  \setlength{\itemsep}{0pt}\setlength{\parskip}{0pt}}\usepackage{longtable,booktabs,array}
\usepackage{calc} % for calculating minipage widths
% Correct order of tables after \paragraph or \subparagraph
\usepackage{etoolbox}
\makeatletter
\patchcmd\longtable{\par}{\if@noskipsec\mbox{}\fi\par}{}{}
\makeatother
% Allow footnotes in longtable head/foot
\IfFileExists{footnotehyper.sty}{\usepackage{footnotehyper}}{\usepackage{footnote}}
\makesavenoteenv{longtable}
\usepackage{graphicx}
\makeatletter
\def\maxwidth{\ifdim\Gin@nat@width>\linewidth\linewidth\else\Gin@nat@width\fi}
\def\maxheight{\ifdim\Gin@nat@height>\textheight\textheight\else\Gin@nat@height\fi}
\makeatother
% Scale images if necessary, so that they will not overflow the page
% margins by default, and it is still possible to overwrite the defaults
% using explicit options in \includegraphics[width, height, ...]{}
\setkeys{Gin}{width=\maxwidth,height=\maxheight,keepaspectratio}
% Set default figure placement to htbp
\makeatletter
\def\fps@figure{htbp}
\makeatother

\KOMAoption{captions}{tableheading}
\makeatletter
\@ifpackageloaded{caption}{}{\usepackage{caption}}
\AtBeginDocument{%
\ifdefined\contentsname
  \renewcommand*\contentsname{Table of contents}
\else
  \newcommand\contentsname{Table of contents}
\fi
\ifdefined\listfigurename
  \renewcommand*\listfigurename{List of Figures}
\else
  \newcommand\listfigurename{List of Figures}
\fi
\ifdefined\listtablename
  \renewcommand*\listtablename{List of Tables}
\else
  \newcommand\listtablename{List of Tables}
\fi
\ifdefined\figurename
  \renewcommand*\figurename{Figure}
\else
  \newcommand\figurename{Figure}
\fi
\ifdefined\tablename
  \renewcommand*\tablename{Table}
\else
  \newcommand\tablename{Table}
\fi
}
\@ifpackageloaded{float}{}{\usepackage{float}}
\floatstyle{ruled}
\@ifundefined{c@chapter}{\newfloat{codelisting}{h}{lop}}{\newfloat{codelisting}{h}{lop}[chapter]}
\floatname{codelisting}{Listing}
\newcommand*\listoflistings{\listof{codelisting}{List of Listings}}
\makeatother
\makeatletter
\makeatother
\makeatletter
\@ifpackageloaded{caption}{}{\usepackage{caption}}
\@ifpackageloaded{subcaption}{}{\usepackage{subcaption}}
\makeatother
\ifLuaTeX
  \usepackage{selnolig}  % disable illegal ligatures
\fi
\usepackage{bookmark}

\IfFileExists{xurl.sty}{\usepackage{xurl}}{} % add URL line breaks if available
\urlstyle{same} % disable monospaced font for URLs
\hypersetup{
  pdftitle={FAQ},
  colorlinks=true,
  linkcolor={blue},
  filecolor={Maroon},
  citecolor={Blue},
  urlcolor={Blue},
  pdfcreator={LaTeX via pandoc}}

\title{FAQ}
\author{}
\date{}

\begin{document}
\maketitle

\renewcommand*\contentsname{Table of contents}
{
\hypersetup{linkcolor=}
\setcounter{tocdepth}{3}
\tableofcontents
}
\section{When and where is class?}\label{when-and-where-is-class}

\begin{itemize}
\tightlist
\item
  When: Monday, Wednesday, and Friday, 1:10PM-2:30PM
\item
  Where: Carnegie 225
\end{itemize}

\section{Where do I get my textbook?}\label{where-do-i-get-my-textbook}

The textbook for this class is ``Quantum Mechanics: A Paradigms
Approach, 1st edition'' by David McIntyre (Cambridge University Press).
The library should have copies available, but you can also purchase it
at the Bates bookstore.

To find the library copies, go to \url{librarysearch.bates.edu} and
search for the course number or my name. Write down the book's call
number and ask for it at the library.

\includegraphics[width=2.08333in,height=\textheight]{images/QMMcIntyre.jpg}

\section{When are your office hours?}\label{officehours}

Office hours:

\begin{itemize}
\tightlist
\item
  Mondays 11AM-12PM
\item
  Wednesdays 4PM-5PM
\end{itemize}

I will hold my office hours in my research space (details in Lyceum)

\section{Should I go to office hours?}\label{goingtoofficehours}

Office hours are my open hours, when I have specifically set aside time
to talk to my students (that's you!). Please always feel free to come
by. You can ask questions about homework, concepts from class, projects,
careers in physics, navigating the major, or anything else you want to
ask. I will prioritize questions about course content, but I'm always
happy to talk about other questions you might have.

\section{Where do I find assignment instructions/course
handouts/slides?}\label{findstuff}

I created this website to make the syllabus and basic questions about
the course easy, but not everything will be here. Handouts, in-class
activities, specific assignments, and class slides and/or recordings
will all be in the course page on Lyceum.

If you are enrolled in the class but do not have access to Lyceum,
please let me know immediately so I can add you.

\section{When is \_\_ due?}\label{duedates}

You can find all due dates on the
\href{https://docs.google.com/spreadsheets/d/12bIFUWPCjQ1M-Uhela3GO7mmHF07_UZhAoRJAUmq4t4/edit?usp=sharing}{course
schedule}. Dates are subject to change, but will always be kept up to
date on the schedule so check back if you're not sure.

\section{Can I have an extension?}\label{extensionpolicy}

In general, I am happy to work with you on deadlines, but it's important
for you to communicate with me. You can request extensions using
\href{https://forms.gle/eFx7y7FoSdoukKGC6}{this form}

Please note that the 48 hour extension is automatically granted, while
anything longer than that requires that we have a conversation. That
conversation starts by you proposing a new deadline in the form, and
then I can talk to you about what's possible or reasonable given the
course schedule, your circumstances, and other factors that may vary as
the semester progresses.

For more on my deadlines and extension policy, see the
\href{syllabus.qmd\#deadlines}{syllabus}.

\section{When will you reply to my email?}\label{emails}

Timely communication is really important, but in order to be an
effective instructor and a well-adjusted human, I also need to be able
to take breaks from email.

If you email me on a weekday, you can expect a response from me within
24 hours. If you email me on a weekend or a holiday, you can expect a
response from me by class time on the first day back.

\section{Do you grade on a curve?}\label{gradingcurve}

Grades are not curved; your grade depends only on your own performance,
supporting your fellow students will help every one of you.

\section{Can I use ChatGPT in this class?}\label{ChatGPT}

Large Language Models (LLMs) are likely here to stay. Banning them from
the classroom is both pointless and actively unhelpful. If you are going
to use ChatGPT or a similar LLM in this class, I only require two
things:

\begin{enumerate}
\def\labelenumi{\arabic{enumi}.}
\item
  That you do a little reading on Large Language Models. Specifically,
  read
  \href{https://stackoverflow.blog/2023/07/03/do-large-language-models-know-what-they-are-talking-about/}{this
  article about how LLMs actually work} and
  \href{https://montrealethics.ai/what-lies-behind-agi-ethical-concerns-related-to-llms/}{this
  article about the ethical considerations around LLMs}
\item
  That you consider ChatGPT a source that you must cite. If you use
  ChatGPT or another LLM to help with any assignment, you \textbf{must}
  acknowledge that help and give a short (one-sentence) description of
  what the LLM did for you and how it helped you solve the problem.
\end{enumerate}

Remember that LLMs are designed to predict the most probable response to
a question, which is not always the best or even correct response. They
are prone to ``hallucinating'' (making stuff up, often in a way that is
convincing but still false). They can't do creative problem solving,
which is the most important skill we are teaching in this class.

\section{What should I do if I get behind on my work?}\label{catchingup}

First, reach out to me. It helps me to know that you are working on
catching up. I am also happy to meet and help you set adjusted deadlines
to get back on track.

Second, if you are feeling really overwhelmed, consider scheduling an
appointment with
\href{https://www.bates.edu/student-academic-support-center/learning-strategies/}{a
learning strategies tutor} at the Student Academic Support Center. Their
role is to support students seeking help with time management,
organization, reading, test-taking, note-taking, and other academic
skills. They can help you talk through what strategies work for you,
what strategies don't, and how to manage your time and energy in a more
sustainable way.

\section{I have academic accommodations. What should I
do?}\label{accommodations}

I have tried to bring the concept of
\href{https://www.washington.edu/doit/what-universal-design-0}{universal
design} into how I have planned and structured this course, so I hope
that any accommodations you have are already built into this course
(aside from extra time on tests, which can't be accommodated during
class times, but you can take the tests through
\href{https://www.bates.edu/accessible-education-student-support/requesting-services/how-to-register-for-accommodations/}{Accessible
Education and Student Support}. However, I don't expect to have done
this perfectly (as there is no such thing), so if you have need of
certain accommodations that are not already provided by this class,
please let me know and I will do my best to meet those needs.

\section{What if I need more support in this class?}\label{extrahelp}

The Student Academic Support Center (SASC) has lots of tutors who can
help you strengthen your math skills, problem solving skills, and study
skills. Please reach out to them to see how they can help, in addition
to coming to my office hours, where I am more than happy to walk through
problems with you.



\end{document}
